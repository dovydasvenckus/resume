%%%%%%%%%%%%%%%%%%%%%%%%%%%%%%%%%%%%%%%%%
% Medium Length Professional CV
% LaTeX Template
% Version 2.0 (8/5/13)
%
% This template has been downloaded from:
% http://www.LaTeXTemplates.com
%
% Original author:
% Trey Hunner (http://www.treyhunner.com/)
%
% Important note:
% This template requires the resume.cls file to be in the same directory as the
% .tex file. The resume.cls file provides the resume style used for structuring the
% document.
%
%%%%%%%%%%%%%%%%%%%%%%%%%%%%%%%%%%%%%%%%%

%----------------------------------------------------------------------------------------
%	PACKAGES AND OTHER DOCUMENT CONFIGURATIONS
%----------------------------------------------------------------------------------------

\documentclass[]{resume} % Use the custom resume.cls style

\usepackage[left=0.75in,top=0.6in,right=0.75in,bottom=0.6in]{geometry} % Document margins
\usepackage[utf8x]{inputenc}
\def\LTfontencoding{L7x}
\usepackage[\LTfontencoding]{fontenc}
\usepackage[lithuanian]{babel}

\name{Dovydas Venckus} % Your name
\address{Vilnius \\ Musninkų 14-1} % Your address
\address{(+370)~$\cdot$~67802390 \\ dovydas.venck@gmail.com} % Your phone number and email

\begin{document}

%----------------------------------------------------------------------------------------
%	EDUCATION SECTION
%----------------------------------------------------------------------------------------

\begin{rSection}{Išsilavinimas}

{\bf Kretingos Jurgio Pabrėžos gimnazija} \hfill {\em 1999-2011} \\
{\bf Vilniaus Universitetas} \hfill {\em 2011-dabar(4-tame kurse)} \\
Matematikos ir informatikos fakultetas\\
Programų sistemų bakalauras\\

\end{rSection}

%----------------------------------------------------------------------------------------
%	WORK EXPERIENCE SECTION
%----------------------------------------------------------------------------------------
\begin{rSection}{Darbo patirtis}
Dirbau metus įmonėje "Dekbera" jaunesniuoju Java programuotoju.
Pagrinde dirbau prie internetinių aplikacijų, naudojant Grails karkasą.
Be darbo su Grails teko parašyti keletą internetinių SOAP servisų naudojant Spring ir Apache CXF.
Šiuos servisus teko testuoti su JUnit biblioteka.
\end{rSection}

%----------------------------------------------------------------------------------------
%   SHORT DESCRIPTION
%----------------------------------------------------------------------------------------
\begin{rSection}{Trumpas aprašymas}

Turiu 1 metų darbinę patirtį su Java, Groovy, Grails ir Spring technologijomis.
Darbe daugiausiai teko dirbti su Grails karkasu. Todėl mano Spring žinios yra žemesnio
lygio nei Grails. Tačiau Spring karkasu domiuosi laisvalaikiu ir naudoju savo projektuose.

Laisvalaikiu domiuosi naujomis technologijomis ir manau, kad nebūtų didelė problema
išmokti technologijas, kurių dar nėra tekę naudoti.
\end{rSection}

%----------------------------------------------------------------------------------------
%	TECHNICAL STRENGTHS SECTION
%----------------------------------------------------------------------------------------

\begin{rSection}{Asmeniniai sugebėjimai}

\begin{tabular}{ @{} >{\bfseries}l @{\hspace{6ex}} l }
Kalbos & Lietuvių, Anglų\\
Programavimo kalbos & Java, Groovy, Python, Ruby, C++  \\
Karkasai & Grails, Spring \\
Front-end & HTML, CSS, Javascript, JQuery, Knockout.js, Google Maps API, DataTables \\
Duomenų bazės & PostgreSQL, MySql, Oracle \\
Įrankiai & Git, Mercurial, Gradle, Maven, Vim, Eclipse, Intellij IDEA, Netbeans \\
Kitos technologijos & JPA, Hibernate, Apache-CXF, Tomcat, JUnit \\
Operacinės sistemos & Linux, Windows
\end{tabular}

\end{rSection}

%----------------------------------------------------------------------------------------
%	EXAMPLE SECTION
%----------------------------------------------------------------------------------------

%\begin{rSection}{Section Name}

%Section content\ldots

%\end{rSection}

%----------------------------------------------------------------------------------------

\end{document}
