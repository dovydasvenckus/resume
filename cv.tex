%%%%%%%%%%%%%%%%%%%%%%%%%%%%%%%%%%%%%%%%%
% Medium Length Professional CV
% LaTeX Template
% Version 2.0 (8/5/13)
%
% This template has been downloaded from:
% http://www.LaTeXTemplates.com
%
% Original author:
% Trey Hunner (http://www.treyhunner.com/)
%
% Important note:
% This template requires the resume.cls file to be in the same directory as the
% .tex file. The resume.cls file provides the resume style used for structuring the
% document.
%
%%%%%%%%%%%%%%%%%%%%%%%%%%%%%%%%%%%%%%%%%

%----------------------------------------------------------------------------------------
%	PACKAGES AND OTHER DOCUMENT CONFIGURATIONS
%----------------------------------------------------------------------------------------

\documentclass[]{resume} % Use the custom resume.cls style

\usepackage[left=0.75in,top=0.6in,right=0.75in,bottom=0.6in]{geometry} % Document margins
\usepackage[utf8x]{inputenc}
\def\LTfontencoding{L7x}
\usepackage[\LTfontencoding]{fontenc}
\usepackage[lithuanian]{babel}

\name{Dovydas Venckus} % Your name
\address{Vilnius \\ Musninkų 14-1} % Your address
\address{(+370)~$\cdot$~67802390 \\ dovydas.venck@gmail.com} % Your phone number and email

\begin{document}

%----------------------------------------------------------------------------------------
%	EDUCATION SECTION
%----------------------------------------------------------------------------------------

\begin{rSection}{Išsilavinimas}

{\bf Kretingos Jurgio Prabrėžos gimnazija} \hfill {\em 1999-2011} \\ 
{\bf Vilniaus Universitetas} \hfill {\em 2011-dabar(4-tame kurse)} \\ 
Matematikos ir informatikos fakultetas\\
Programų sistemų bakalauras\\

\end{rSection}

%----------------------------------------------------------------------------------------
%	WORK EXPERIENCE SECTION
%----------------------------------------------------------------------------------------
\begin{rSection}{Darbo patirtis}
Dirbau pusę metų įmonėje "Dekbera" jaunesniuoju java/groovy programuotoju.
Pagrinde dirbau prie web aplikacijos, kuri skirta Lietuvos greitūjų dispečeriams.
Ši sistema leidžia stebėti greitosios pagalbos automobilų informaciją. 

Šiame darbe naudojau grails frameworką, taip pat teko išmėginti konckout.js biblioteką. 
Teko nemažai padirbėti su Google maps API: judančias objektais žemėlapyje, heatmap žemėlapiu.
\end{rSection}
%----------------------------------------------------------------------------------------
%   SHORT DESCRIPTION
%----------------------------------------------------------------------------------------
\begin{rSection}{Trumpas aprašymas}

Turiu pusės metų darbinę patirtį su java, groovy ir grails frameworku.
Taip pat turiu poros metų patirtį dirbant linux operacinėję sistemoje. 
Truputį nusimanau ir apie linux serverių administravimą. 
Yra teke naudoti Spring frameworką, savo projektams.

Laisvu laiku domiuosi naujomis technologijomis, rašau skriptus, mėginu automatizuoti įkyrius darbus.

Esu atviras naujoms technologijoms ir iššūkiams. Žinoma, kol tie iššūkiai proto ribose.
 
\end{rSection}

%----------------------------------------------------------------------------------------
%	TECHNICAL STRENGTHS SECTION
%----------------------------------------------------------------------------------------

\begin{rSection}{Asmeniniai sugebėjimai}

\begin{tabular}{ @{} >{\bfseries}l @{\hspace{6ex}} l }
Kalbos & Lietuvių, Anglų\\
Programavimo kalbos & Java, Groovy, Python, Ruby, C++  \\
Frameworkai & Grails, Spring \\
Front-end & HTML, CSS, Javascript, JQuery, knockout.js \\
Duomenų bazės & PostgreSQL, MySql, Oracle \\
Įrankiai & Git, Mercurial, Gradle, Maven, Vim, Eclipse, Intellij IDEA, Netbeans \\
Kitos technologijos & JPA, Hibernate, Tomcat, JUnit, Google maps API \\
Operacinės sistemos & Linux, Windows
\end{tabular}

\end{rSection}

%----------------------------------------------------------------------------------------
%	EXAMPLE SECTION
%----------------------------------------------------------------------------------------

%\begin{rSection}{Section Name}

%Section content\ldots

%\end{rSection}

%----------------------------------------------------------------------------------------

\end{document}
